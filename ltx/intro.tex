Probabilistic programming languages have grown increasingly popular in recent years because they allow for the concise definition of complex statistical models. They also provide tools for sampling the (usually Bayesian) models. 

YAPPL is inspired by the probabilistic programming language Church, an implementation of a pure subset of Scheme (a dialect of Lisp) for generating models using probabilistic functions. Church relies on the standard Lisp syntax, which is unintuitive and difficult to read. The syntax of YAPPL is inspired by OCaml and contains special constructs for the probabilistic elements of the language, which makes it more approachable and human-readable than Church. 

In particular, the goal of YAPPL is to make it easy to define generative Bayesian models. That is, to define a probabilistic model and then sample the parameters {\em conditionally} from that distribution based on observed samples drawn from that model. We accomplish this by building the conditional sampling and another useful construct called memoization directly into the language. A memoized function remembered what value it returned for previously evaluated parameter values and always returns those values in the future.  

YAPPL is an almost pure functional programming language. The only functions that produce side-effects are explicitly built into the language: printing, seeding the random number generator, generating random numbers, and functions defined via memoization. 